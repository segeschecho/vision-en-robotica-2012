\documentclass[12pt,a4paper]{article}
\usepackage[spanish] {babel}
\usepackage[latin1]{inputenc} %para que me tome los acentos de una sin el /'
\usepackage{theorem} %para los teoremas
\usepackage{graphicx}%para los graficos
\usepackage{color} % para los graficos con letra latex
\usepackage{transparent} % para los graficos con letra latex
\usepackage{amsfonts} % para las letras de los conjuntos reales naturales enteros
\usepackage{float} %para poder decir donde quiero que esten las figuras
\usepackage{amsmath} %para poner el numero de capitulo antes de cada figura
\usepackage{fancyhdr} %para la raya superior, la cabecera
\usepackage{lastpage} %numero de la ultima pagina lo utilice en el estilo fancy
\usepackage{url} %para la bibliografia de URLs
\usepackage{fancyvrb} %para decorar codigo verbatim
\usepackage{float} %me permite poner caption en codigo
\usepackage{placeins} %define \FloatBarrier para que los "Floats"' como las imagenes no pasen una seccion o subseccion (es una barrera para los floats, como el nombre lo indica)
\usepackage[scanall]{psfrag} % para que aparesca la letra latex en los graficos
\usepackage{courier} %el codigo verbatim se ve mucho mejor
\usepackage{afterpage}
\usepackage{stmaryrd}   
\usepackage{fourier}    %para poder usar operadores matematicos y el simbolo \varpartialdiff
\usepackage{amssymb}   % para poder usar el conjunto vacio

%\floatstyle{ruled}
%\newfloat{Codigo}{ht}{lop}%[chapter]
%\fvset{fontsize=\small}
%\floatname{Codigo}{Código} %para que aparesca Código en el caption y no Codigo
%
%\fancyhf{}
%\fancyhead{}
%\fancyfoot{} % clear all footer fields
%\headheight = 10pt %16.5pt
%\pagestyle{fancy}
%\fancyhead[LE]{\slshape \leftmark}
%\fancyhead[RO]{\slshape \rightmark}
%\fancyhead[C]{}
%\providecommand{\e}[1]{\ensuremath{\times 10^{#1}}}
%\fancyfoot[C]{\thepage}
% \headsep = 0.4cm
% \headwidth = 16cm
% %\hoffset = 0.3 cm
% \marginparwidth=0cm
% \marginparsep=0cm
% \linespread{1.4}
% \oddsidemargin=0,4cm
% \evensidemargin=-0,6cm
% \textwidth = 16cm
% \footskip = 60pt


%%%%%%%%%%%%%%%ANCHO DE MARGENES Y TAMAÑO DE TEXTO%%%%%%%%%%%%%%%%%%%%%%%%%%%%%%

\setlength{\oddsidemargin}{0in}
\setlength{\textwidth}{6.75in}
\setlength{\topmargin}{-0.5in}
\setlength{\headheight}{0in}
\setlength{\textheight}{9.5in}

%\setlength{\oddsidemargin}{0,4cm}
%\setlength{\evensidemargin}{-0,6cm}
%\setlength{\textwidth}{17cm}
%%%%%%%%%%%%%%%%%%%%%%%%%%%%%%%%%%%%%%%%%%%%%

\begin{document}

\begin{center}
{\bf Resumen Visi\'on en Rob\'otica}

\medskip
Taih\'u Pire, Emiliano Gonz\'alez, Sergio Gonzalez

Junio, 2012
\end{center}


\setlength{\parindent}{0pt} %elimina la sangria

\section{Espacio proyectivo: puntos, rectas y c\'onicas}

\begin{itemize}
	\item $x \in l \Leftrightarrow x^{\top}l = l^{\top} x = 0$

	\item $l_{1} // l_{2} \Rightarrow l_{1} \times l_{2} = \left[ {\begin{smallmatrix}
	 x_{1}\\
	 x_{2}\\
	 0\\
	\end{smallmatrix} } \right]
	$

	\item Linea tangente a la c\'onica: $C x = l$

	\item $C^{*} l = x$

	\item $x \in C \Leftrightarrow x^{\top} C x = 0$


	\item $l \in C^{*} \Leftrightarrow l^{\top} C^{*} l = 0$

	\item $ \exists C^{-1} \Rightarrow C^{*} = C^{-1}$

	\item $ \nexists C^{-1} \Rightarrow C_{ij}^{*} = (-1)^{ij} det(\hat{C}_{ij}) \quad i=1\dots n$

	\item $C = lm^{\top}+ml^{\top}$ la matriz C tiene rango 2 (es degenerada).

	\item $C^{*} = xy^{\top}+yx^{\top}$

	\item $(C^{*})^{*}  \neq C$ si $C$ es singular.

	\item $ax^{2}+bxy+cy^{2}+dx+ey+f = 0$

	\item $ax_{1}^{2}+bx_{1}x_{2}+cx_{2}^{2}+dx_{1}x_{3}+ex_{2}x_{3}+fx_{3}^{2} = 0$


	\item $
	C =
	\left[ {\begin{smallmatrix}
	 a & b/2 & d/2 \\
	 b/2 & c & e/2 \\
	 d/2 & e/2 & f
	\end{smallmatrix} } \right]
	$
\end{itemize}

\section{Transformaciones Proyectivas}

{\bf Transformaci\'on:} $h(x)= Hx$. Donde $H$ es {\bf no singular}.

Transformaci\'on de Puntos, Lineas y C\'onicas:

\begin{itemize}
	\item $x'= H x$

	\item $l'= H^{-\top} l$

	\item $C'= H^{-\top} C H^{-1}$

	\item ${C^{*}}'= H C^{*} H^{\top}$
\end{itemize}

\subsection{Tipos de transformaciones}

\begin{itemize}
	\item {\bf Isom\'etrica:} $
	H_{E} =
	\left[ {\begin{smallmatrix}
	 \epsilon \cos \theta & -\sin \theta& t_{x} \\
	 \epsilon \sin \theta & \cos \theta& t_{y} \\
	 0 & 0 & 1
	\end{smallmatrix} } \right]
	\quad \epsilon = \pm 1$	\hspace*{1cm}(Grados de libertad: 3)
	
	{\bf Invariantes:} 
	\begin{itemize}
		\item Longitud
		\item \'area
	\end{itemize}

	\item {\bf Similaridad:}$
	H_{S} =
	\left[ {\begin{smallmatrix}
	 s \cos \theta & - s \sin \theta& t_{x} \\
	 s \sin \theta & s\cos \theta& t_{y} \\
	 0 & 0 & 1
	\end{smallmatrix} } \right]
	$ \hspace*{1cm}(Grados de libertad: 4)
	
	{\bf Invariantes:} 
	\begin{itemize}
		\item raz\'on de longitudes
		\item \'angulo
		\item Puntos Circulares: {\bf I}, {\bf J}
	\end{itemize}

	\item {\bf Afinidad:}$
	H_{A} =
	\left[ {\begin{smallmatrix}
	 A & {\bf t}\\
	 {\bf 0}^{\top} & 1 \\
	\end{smallmatrix} } \right]
	$ donde $A$ es {\bf no singular}  \hspace*{1cm}(Grados de libertad: 6)

	\begin{enumerate}
		\item $A = R(\theta)R(-\phi)DR(\phi)$ con $D = \left[ {\begin{smallmatrix}
		 \lambda_{1} & 0\\
		 0 & \lambda_{2}\\
		\end{smallmatrix} } \right]$

		\item $
		H_{A}^{-1} =
		\left[ {\begin{smallmatrix}
		 A^{-1} & -A^{-1}{\bf t}\\
		 {\bf 0}^{\top} & 1 \\
		\end{smallmatrix} } \right]
		$

		\item $l_{\infty} = \left[ {\begin{smallmatrix}
		 0\\
		 0\\
		 1\\
		\end{smallmatrix} } \right]$

		\item $l_{\infty} = H_{A} l_{\infty}$

		\item $x \in l \Rightarrow H_{A}(x) \in H_{A}(l)$

		\item $l_{1} // l_{2} \Rightarrow H_{A}(l_{1})//H_{A}(l_{2})$
	\end{enumerate}
	
	{\bf Invariantes:} 
	\begin{itemize}
		\item Paralelismo
		\item Raz\'on de \'areas
		\item Raz\'on de longitudes en lineas paralelas o colineales
		\item $l_{\infty}$
	\end{itemize}

\item {\bf Proyectiva:}
$H_{P} =
\left[ {\begin{smallmatrix}
 A & {\bf t}\\
 {\bf v}^{\top} & v \\
\end{smallmatrix} } \right]
$ \hspace*{1cm}(Grados de libertad: 8)

	\begin{enumerate}
		\item $\left[ {\begin{smallmatrix}
		 {\bf A} & {\bf t}\\
		 {\bf 0}^{\top} & 1 \\
		\end{smallmatrix} } \right] \left[ {\begin{smallmatrix}
		 x_{1}\\
		 x_{2}\\
		 0\\
		\end{smallmatrix} } \right] = \left[ {\begin{smallmatrix}
		 A\left[ {\begin{smallmatrix}
		  x_{1}\\
		   x_{2}\\
		 \end{smallmatrix} } \right]\\
		 0\\
		\end{smallmatrix} } \right]
		$

		\item $\left[ {\begin{smallmatrix}
		  A & {\bf t}\\
		 {\bf v}^{\top} & v \\
		\end{smallmatrix} } \right] \left[ {\begin{smallmatrix}
		 x_{1}\\
		 x_{2}\\
		 0\\
		\end{smallmatrix} } \right] = \left[ {\begin{smallmatrix}
		 A\left[ {\begin{smallmatrix}
		  x_{1}\\
		   x_{2}\\
		 \end{smallmatrix} } \right]\\
		 v_{1}x_{1}+v_{2}x_{2}\\
		\end{smallmatrix} } \right]
		$

		\item $H = H_{S}H_{A}H_{P} = \left[ {\begin{smallmatrix}
		 sR & {\bf t}\\
		 {\bf 0}^{\top} & 1 \\
		\end{smallmatrix} } \right]\left[ {\begin{smallmatrix}
		 K & {\bf 0}\\
		 {\bf 0}^{\top} & 1 \\
		\end{smallmatrix} } \right]\left[ {\begin{smallmatrix}
		 I & {\bf 0}\\
		 {\bf v}^{\top} & v \\
		\end{smallmatrix} } \right]=\left[ {\begin{smallmatrix}
		 A & {\bf t}\\
		 {\bf v}^{\top} & v \\
		\end{smallmatrix} } \right]
		$
	\end{enumerate}
	
	{\bf Invariantes:} 
	\begin{itemize}
		\item Cross Ratio
	\end{itemize}

\end{itemize}

\section{Propiedades m\'etricas de las im\'agenes}

\begin{enumerate}
	\item $Cross(x_{1}, x_{2}, x_{3}, x_{4}) = \dfrac{|x_{1},x_{2}||x_{3},x_{4}|}{|x_{1},x_{3}||x_{2},x_{4}|}$ donde $|x_{i},x_{j}| = det \left[ {\begin{smallmatrix}
	 x_{i1} & x_{j1}\\
	 x_{i2} & x_{j2} \\
	\end{smallmatrix} } \right]$ $x_{i} \in P^{1}$

\item $Cross(h(x_{1}), h(x_{2}), h(x_{3}), h(x_{4})) = Cross(x_{1}, x_{2}, x_{3}, x_{4})$ donde $h:P^{1} \rightarrow P^{1}$
\end{enumerate}

{\bf Puntos circulares:}

\begin{itemize}
	\item $I = \left[ {\begin{smallmatrix}
	 1\\
	 i\\
	 0\\
	\end{smallmatrix} } \right]$
	$J = \left[ {\begin{smallmatrix}
	 1\\
	 -i\\
	 0\\
	\end{smallmatrix} } \right]$

	\item $I' = H_{S}I = se^{-i\theta}I = I$

	\item ${C^{*}_{\infty}}' = {\bf IJ}^{\top} + {\bf JI}^{\top}$ degenerada de rango 2

	\item $
	{C^{*}_{\infty}} =
	\left[ {\begin{smallmatrix}
	 1 & 0 & 0 \\
	 0 & 1 & 0 \\
	 0 & 0 & 0
	\end{smallmatrix} } \right]
	$

	\item %${C^{*}_{\infty}}' = H_{S}{C^{*}_{\infty}}H_{S}^{\top} = {C^{*}_{\infty}}$

	\item $H({C^{*}_{\infty}}) = {C^{*}_{\infty}}$

	\item ${C^{*}_{\infty}} l_{\infty} = 0$

	\item $\cos{\theta} = \dfrac{{\bf l}^{\top} {C^{*}_{\infty}} {\bf m}}
	{\sqrt{({\bf l}^{\top} {C^{*}_{\infty}} {\bf l})({\bf m}^{\top} {C^{*}_{\infty}} {\bf m}) }}$ \hspace{2cm} ${\bf l}$ $\bot$ ${\bf m} \Rightarrow {\bf l}^{\top} {C^{*}_{\infty}} {\bf m} = 0$

\end{itemize}

\section{C\'amaras}

\begin{itemize}

	\item $R = R_{Z}(\phi_{z})R_{Y}(\phi_{y})R_{X}(\phi_{x})$

\item $
	R_{Z}(\phi_{z}) =
	\left[ {\begin{smallmatrix}
	 \cos{\phi_{z}} & -\sin{\phi_{z}} & 0 \\
	 \sin{\phi_{z}} & \cos{\phi_{z}} & 0 \\
	 0 & 0 & 1\\
	\end{smallmatrix} } \right] \quad R_{Y}(\phi_{y}) =
	\left[ {\begin{smallmatrix}
		\cos{\phi_{y}} & 0 & \sin{\phi_{y}}\\
		 0 & 1 & 0\\
	    -\sin{\phi_{y}} & 0 & \cos{\phi_{y}}\\
	\end{smallmatrix} } \right] \quad R_{X}(\phi_{x}) =
	\left[ {\begin{smallmatrix}
		1 & 0 & 0\\ 
	    0 & \cos{\phi_{x}} & -\sin{\phi_{x}}\\
	    0 & \sin{\phi_{x}} & \cos{\phi_{x}}\\
	\end{smallmatrix} } \right]
	$

	\item $C = (\overline{C}, 1)$.

	\item $P = KR [I |- \overline{C}] = M[I|M^{-1}p_{4}] = [M|p_{4}]$ donde $M= KR$ y $p_{4}$ la \'ultima columna de $P$

	\item $M$ no singular (camara finita) $\Rightarrow C = \left[ {\begin{smallmatrix}
		-M^{-1}p_{4}\\ 
		 1\\
	\end{smallmatrix} } \right]$ 

	\item $M$ singular (camara en el infinito) $\Leftrightarrow C = \left[ {\begin{smallmatrix}
		{\bf d}\\ 
		0\\
	\end{smallmatrix} } \right]$ donde $M {\bf d} = 0$. Entonces el centro de la camara es un punto en el infinito.

	\item $
	K = \left[ {\begin{smallmatrix}
		\alpha_{x} & s & p_{x}\\ 
	    0 & \alpha_{y} & p_{y}\\
	    0 & 0 & 1\\
	\end{smallmatrix} } \right]
	$ donde $\alpha_{x} = f m_{x}$, $\alpha_{y} = f m_{y}$. $p_{x}$, $p_{y}$ es el punto principal y $s$ es el skew.

\end{itemize}


\section{Geometr\'ia epipolar}

\begin{itemize}
	\item $[a]_{x} = \left[ {\begin{smallmatrix}
		0 & -a_3 & a_2\\ 
		a_3  & 0 & -a_1\\
		-a_2 & a_1 & 0
		\end{smallmatrix} } \right] $

	\item $[a]_{x}M = M^{-\top}[M^{-1}a]_{x}$

	\item $l' = Fx = [e']_{x} H_{\pi} x$ $\implies$ $F = [e']_{x} H_{\pi}$

	\item $F = [e']_{x} P' P^{+}$   \hspace*{1cm}(Grados de libertad: 7)

	\item $P^{+} = P^{\top}(PP^{\top})^{-1}$

	\item ${x'}{}^{\top}Fx =0$ para toda correspondencia $\{x \leftrightarrow x' \}$

	\item $l = F^{\top}x' \implies F' = F^{\top}$

	\item $Fe = 0$ y $F^{\top}e' = 0$

	\item $PC = 0$. Donde $C \in P^{3}$ es el centro de la camara.

	\item $e = PC'$ y $e' = P'C$

	\item Dadas $\{x \leftrightarrow x' \}$ se obtienen  $\{\hat{x} \leftrightarrow \hat{x}' \} \diagup \hat{x} = Hx$, $\hat{x}'=Hx'  \implies \hat{F} = H^{-\top}FH^{-1}$
	
	\item $x_{0} = M{\bf m^{3}}$ donde $x_{0}$ es el punto principal y ${\bf m^{3}}$ es la tercer fila de la matriz $M$.
	
	\item ${\bf v}= det(M){\bf m^{3}}$ donde ${\bf v}$ es el vector direcci\'on del eje principal de la c\'amara hacia el plano de la imagen.

\end{itemize}

\end{document}
